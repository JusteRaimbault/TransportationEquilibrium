% Template for Elsevier CRC journal article
% version 1.2 dated 09 May 2011

% This file (c) 2009-2011 Elsevier Ltd.  Modifications may be freely made,
% provided the edited file is saved under a different name

% This file contains modifications for Transportation Research Procedia

% Changes since version 1.1
% - added "procedia" option compliant with ecrc.sty version 1.2a
%   (makes the layout approximately the same as the Word CRC template)
% - added example for generating copyright line in abstract

%-----------------------------------------------------------------------------------

%% This template uses the elsarticle.cls document class and the extension package ecrc.sty
%% For full documentation on usage of elsarticle.cls, consult the documentation "elsdoc.pdf"
%% Further resources available at http://www.elsevier.com/latex

%-----------------------------------------------------------------------------------

%%%%%%%%%%%%%%%%%%%%%%%%%%%%%%%%%%%%%%%%%%%%%%%%%%%%%%%%%%%%%%
%%%%%%%%%%%%%%%%%%%%%%%%%%%%%%%%%%%%%%%%%%%%%%%%%%%%%%%%%%%%%%
%%                                                          %%
%% Important note on usage                                  %%
%% -----------------------                                  %%
%% This file should normally be compiled with PDFLaTeX      %%
%% Using standard LaTeX should work but may produce clashes %%
%%                                                          %%
%%%%%%%%%%%%%%%%%%%%%%%%%%%%%%%%%%%%%%%%%%%%%%%%%%%%%%%%%%%%%%
%%%%%%%%%%%%%%%%%%%%%%%%%%%%%%%%%%%%%%%%%%%%%%%%%%%%%%%%%%%%%%

%% The '3p' and 'times' class options of elsarticle are used for Elsevier CRC
%% The 'procedia' option causes ecrc to approximate to the Word template
\documentclass[3p,times,procedia]{elsarticle}
\flushbottom

%% The `ecrc' package must be called to make the CRC functionality available
\usepackage{ecrc}
%\usepackage{amsmath}


%% The ecrc package defines commands needed for running heads and logos.
%% For running heads, you can set the journal name, the volume, the starting page and the authors

%% set the volume if you know. Otherwise `00'
\volume{00}

%% set the starting page if not 1
\firstpage{1}

%% Give the name of the journal
\journalname{Transportation Research Procedia}

%% Give the author list to appear in the running head
%% Example \runauth{C.V. Radhakrishnan et al.}
\runauth{J. Raimbault}

%% The choice of journal logo is determined by the \jid and \jnltitlelogo commands.
%% A user-supplied logo with the name <\jid>logo.pdf will be inserted if present.
%% e.g. if \jid{yspmi} the system will look for a file yspmilogo.pdf
%% Otherwise the content of \jnltitlelogo will be set between horizontal lines as a default logo

%% Give the abbreviation of the Journal.
\jid{aaspro}

%% Give a short journal name for the dummy logo (if needed)
%\jnltitlelogo{Transportation Research Procedia}

%% Hereafter the template follows `elsarticle'.
%% For more details see the existing template files elsarticle-template-harv.tex and elsarticle-template-num.tex.

%% Elsevier CRC generally uses a numbered reference style
%% For this, the conventions of elsarticle-template-num.tex should be followed (included below)
%% If using BibTeX, use the style file elsarticle-num.bst

%% End of ecrc-specific commands
%%%%%%%%%%%%%%%%%%%%%%%%%%%%%%%%%%%%%%%%%%%%%%%%%%%%%%%%%%%%%%%%%%%%%%%%%%

%% The amssymb package provides various useful mathematical symbols

\usepackage{amssymb}
%% The amsthm package provides extended theorem environments
%% \usepackage{amsthm}

%% The lineno packages adds line numbers. Start line numbering with
%% \begin{linenumbers}, end it with \end{linenumbers}. Or switch it on
%% for the whole article with \linenumbers after \end{frontmatter}.
%% \usepackage{lineno}

%% natbib.sty is loaded by default. However, natbib options can be
%% provided with \biboptions{...} command. Following options are
%% valid:

%%   round  -  round parentheses are used (default)
%%   square -  square brackets are used   [option]
%%   curly  -  curly braces are used      {option}
%%   angle  -  angle brackets are used    <option>
%%   semicolon  -  multiple citations separated by semi-colon
%%   colon  - same as semicolon, an earlier confusion
%%   comma  -  separated by comma
%%   numbers-  selects numerical citations
%%   super  -  numerical citations as superscripts
%%   sort   -  sorts multiple citations according to order in ref. list
%%   sort&compress   -  like sort, but also compresses numerical citations
%%   compress - compresses without sorting
%%
\biboptions{authoryear}

% \biboptions{}

% if you have landscape tables
\usepackage[figuresright]{rotating}
%\usepackage{harvard}
% put your own definitions here:x
%   \newcommand{\cZ}{\cal{Z}}
%   \newtheorem{def}{Definition}[section]
%   ...

% add words to TeX's hyphenation exception list
%\hyphenation{author another created financial paper re-commend-ed Post-Script}


%%%%%%%%%%%%%%%%%%
%% User-defined packages

\usepackage{bbm}

\usepackage[utf8]{inputenc}
\usepackage[T1]{fontenc}


% declarations for front matter

\begin{document}

\begin{frontmatter}

%% Title, authors and addresses

%% use the tnoteref command within \title for footnotes;
%% use the tnotetext command for the associated footnote;
%% use the fnref command within \author or \address for footnotes;
%% use the fntext command for the associated footnote;
%% use the corref command within \author for corresponding author footnotes;
%% use the cortext command for the associated footnote;
%% use the ead command for the email address,
%% and the form \ead[url] for the home page:
%%
%% \title{Title\tnoteref{label1}}
%% \tnotetext[label1]{}
%% \author{Name\corref{cor1}\fnref{label2}}
%% \ead{email address}
%% \ead[url]{home page}
%% \fntext[label2]{}
%% \cortext[cor1]{}
%% \address{Address\fnref{label3}}
%% \fntext[label3]{}

\dochead{19th EURO Working Group on Transportation Meeting, EWGT2016, 5-7 September 2016, Istanbul, Turkey}
%% Use \dochead if there is an article header, e.g. \dochead{Short communication}
%% \dochead can also be used to include a conference title, if directed by the editors
%% e.g. \dochead{17th International Conference on Dynamical Processes in Excited States of Solids}

\title{Investigation Empirique de l'Existence de l'Equilibre Utilisateur Statique}

%% use optional labels to link authors explicitly to addresses:
%% \author[label1,label2]{<author name>}
%% \address[label1]{<address>}
%% \address[label2]{<address>}



%\author[a]{First Author} 
%\author[b]{Second Author}
\author[a,b]{Juste Raimbault\corref{cor1}}

\address[a]{UMR CNRS 8504 G{\'e}ographie-cit{\'e}s, 13 rue du Four, 75006 Paris, France}
\address[b]{UMR-T IFSTTAR 9403 LVMT, Cit{\'e} Descartes, 77455 Champs-sur-Marne, France}

\begin{abstract}
L'Equilibre Utilisateur Statique est un cadre puissant pour l'étude théorique du trafic. Malgré l'hypothèse restreignante de stationnarité des flots qui intuitivement limite son application aux systèmes de trafic réels, de nombreux modèles opérationnels qui l'implémentent sont toujours utilisés sans validation empirique de l'existence de l'équilibre. Nous étudions celle-ci sur un jeu de données de trafic couvrant trois mois sur la région parisienne. L'implémentation d'une application d'exploration interactive de données spatio-temporelles permet de formuler l'hypothèse d'une forte hétérogénéité spatiale et temporelle, guidant les études quantitatives. L'hypothèse de flots localement stationnaires est invalidée en première approximation par les résultats empiriques, comme le montrent une forte variabilité spatio-temporelle des plus courts chemins et des mesures topologiques du réseau comme la centralité de chemin. De plus, le comportement de l'index d'autocorrelation spatiale pour les motifs de congestion à différentes portées spatiales suggère une évolution chaotique à l'échelle locale, en particulier lors des heures de pointe. Nous discutons finalement les implications de ces résultats empiriques et proposons des possibles développements futurs basés sur l'estimation de la stabilité dynamique au sens de Lyapounov des flots de trafic.
\end{abstract}

\begin{keyword}
Equilibre Utilisateur Statique \sep Visualisation de Données Spatio-temporelles \sep Stationnarité Spatio-temporelle \sep Stabilité Dynamique

%% keywords here, in the form: keyword \sep keyword

%% PACS codes here, in the form: \PACS code \sep code

%% MSC codes here, in the form: \MSC code \sep code
%% or \MSC[2008] code \sep code (2000 is the default)

\end{keyword}
\cortext[cor1]{Auteur correspondant. Tel.: +33140464000.% ; fax: +0-000-000-0000.
}
\end{frontmatter}

%\correspondingauthor[*]{Corresponding author. Tel.: +0-000-000-0000 ; fax: +0-000-000-0000.}
\email{juste.raimbault@polytechnique.edu}

%%
%% Start line numbering here if you want
%%
% \linenumbers

%% main text

%\enlargethispage{-7mm}

%%%%%%%%%%%%%%%%%%%%%%%
\section{Introduction}
\label{introduction}


La modélisation du trafic a été largement étudiée depuis les travaux séminaux de Wardrop (\cite{wardrop1952road}) : les enjeux économiques et techniques justifient entre autre le besoin d'une compréhension fine des mécanismes régissant les flots de trafic à différentes échelles. Différentes approches aux objectifs différents coexistent aujourd'hui, parmi lesquels on trouve par exemple les modèles dynamiques de micro-simulation, généralement opposés aux techniques de basant sur l'équilibre. Tandis que la validité des modèles microscopiques a été étudiée de façon conséquente et leur application souvent questionnée, la littérature est relativement pauvre en études empiriques assurant l'hypothèse d'équilibre stationnaire du cadre de l'Equilibre Utilisateur Statique (EUS). De nombreux développements plus réalistes on été documentés dans la littérature, tels l'Equilibre Utilisateur Dynamique Stochastique (EUDS) (voir pour une description par example~\cite{han2003dynamic}). A un niveau intermédiaire entre les cadres statiques et stochastiques se trouve l'Equilibre Utilisateur Stochastique Restreint, pour lequel les choix d'itinéraire des utilisateurs sont contraints à un ensemble d'alternatives réalistes (\cite{rasmussen2015stochastic}). D'autres extensions prenant en compte le comportement de l'utilisateur via des modèles de choix ont été proposé plus récemment, comme~\cite{zhang2013dynamic} qui inclut à la fois l'influence de la tarification routière et de la congestion sur le choix avec un modèle Probit. La relaxation d'autres hypothèses restrictives comme la maximisation pure de l'utilité par l'utilisateur ont aussi été introduites, tels l'Equilibre Utilisateur Borné décrit par~\cite{mahmassani1987boundedly}. Dans ce cadre, l'utilisateur est satisfait si son utilité tombe dans un intervalle et l'équilibre est achevé lorsque chaque utilisateur est satisfait. Les dynamiques résultantes sont plus complexes comme révélé par l'existence d'équilibres multiples, et permet de rendre compte de faits stylisés spécifiques comme des évolutions irréversibles du réseau comme développé par~\cite{guo2011bounded}. D'autres modèles d'attribution de trafic inspirés d'autres domaines ont également été plus récemment proposés: dans~\cite{puzis2013augmented}, une définition étendue de la centralité de chemin qui combine linéairement le centralité des flots non-contraints avec une centralité pondérée par le temps de parcours permet d'obtenir une forte corrélation avec les flots de trafic effectifs, fournissant ainsi un modèle d'attribution de trafic. Cela fournit également des applications pratiques comme l'optimisation de la distribution spatiale des capteurs de trafic.


Malgré ces nombreux développements, de nombreuses études et applications concrètes se reposent toujours sur l'Equilibre Utilisateur Statique. La région parisienne utilise par exemple un modèle statique (MODUS) pour gérer et planifier le trafic. \cite{leurent2014user} introduit un modèle statique de flots qui inclut les recherches locales et le choix du parking : il est légitime de s'interroger, en particulier à de si faibles échelles, si la stationnarité de la distribution des flots est une réalité. Une example d'exploration empirique des hypothèses classiques est donné par~\cite{zhu2010people}, pour lequel les choix d'itinéraires révélés sont étudiés. Les conclusions questionnent le ``premier principe de Wardrop'' qui implique que les utilisateurs choisissent parmi un ensemble d'alternatives parfaitement connu. Dans le même esprit, nous étudions l'existence possible de l'équilibre en pratique. Plus précisément, l'EUS suppose une distribution stationnaire des flots sur l'ensemble du réseau. Cette hypothèse reste valable dans le cas d'une stationnarité locale, tant que l'échelle temporelle d'évolution des paramètres est considérablement plus grande que les échelles typiques de voyage. Le second cas qui est plus plausible et de plus compatible avec les cadres théoriques dynamiques est testé ici.

La suite de ce travail s'organise ainsi : la procédure de collection de données ainsi que le jeu de données sont décrits ; nous présentons ensuite une application interactive pour l'exploration du jeu de données, dans le but de fournir une intuitions sur les motifs présents ; puis nous donnons divers résultats d'analyses quantitatives allant dans le sens d'indices convergents pour une non-stationnarité des flots de trafic ; nous discutons finalement les implications de ces résultats et des développements possibles.




%%%%%%%%%%%%%%%%%%%%%
\section{Collecte des données}


%%%%%%%%%%%%%%%%%%%%%
\subsection{Construction du jeu de données}


Nous proposons de travailler sur l'étude de cas de la région métropolitaine de Paris. Un jeu de données ouvert a été construit, comprenant les liens autoroutiers dans la région, par collecte des données publiques en temps réel des temps de parcours (disponible sur www.sytadin.fr). Comme rappelé par~\cite{bouteiller2013open}, la disponibilité de jeux de données ouverts pour les transports est loin d'être la règle, et nous contribuons ainsi à une ouverture par la construction de notre jeu de données. La procédure de collecte de données consiste en les points suivants, éxecutés toutes les deux minutes par un script \texttt{python} :
\begin{itemize}
\item récupération de la page web brute donnant les informations de trafic
\item parsing du code html afin de récupérer les identifiants des liens de trafic et les temps de parcours correspondants
\item insertion des liens dans une base \texttt{sqlite} avec le temps courant.
\end{itemize}


Our data collection procedure consists in the following simple steps, executed each two minutes by a \texttt{python} script :
\begin{itemize}
\item fetch raw webpage giving traffic information
\item parse html code to retrieve traffic links id and their corresponding travel time
\item insert all links in a \texttt{sqlite} database with the current timestamp.
\end{itemize}
The automatized data collection script continues to enrich the database as time passes, allowing future extensions of this work on a larger dataset and a potential reuse by scientists or planners. The latest version of the dataset is available online (sqlite format) under a Creative Commons License\footnote{at \texttt{http://37.187.242.99/files/public/sytadin{\_}latest.sqlite3}}.


%%%%%%%%%%%%%%%%%%%%%
\subsection{Data Summary}


A time granularity of 2 minutes was obtained for a three months period (February 2016 to April 2016 included). Spatial granularity is in average 10km, as travel times are provided for major links. The dataset contains 101 links. Raw data we use is effective travel time, from which we can construct travel speed and relative travel speed, defined as the ratio between optimal travel time (travel time without congestion, taken as minimal travel times on all time steps) and effective travel time. Congestion is constructed by inversion of a simple BPR function with exponent 1, i.e. we take $c_i = 1 - \frac{t_{i,min}}{t_i}$ with $t_i$ travel time in link $i$ and $t_{i,min}$ minimal travel time.




%%%%%%%%%%%%%%%%%%%%%%
\section{Methods and Results}


%%%%%%%%%%%%%%%%%%%%%%
\subsection{Visualization of spatio-temporal congestion patterns}


As our approach is fully empirical, a good knowledge of existing patterns for traffic variables, and in particular of their spatio-temporal variations, is essential to guide any quantitative analysis. Taking inspiration from an empirical model validation literature, more precisely Pattern-oriented Modeling techniques introduced by~\cite{grimm2005pattern}, we are interested in macroscopic patterns at given temporal and spatial scales: the same way stylized facts are in that approach extracted from a system before trying to model it, we need to explore interactively data in space and time to find relevant patterns and associated scales. We implemented therefore an interactive web-application for data exploration using \texttt{R} packages \texttt{shiny} and \texttt{leaflet}\footnote{source code for the application and analyses is available on project open repository at\\
\texttt{https://github.com/JusteRaimbault/TransportationEquilibrium}}.
It allows dynamical visualization of congestion among the whole network or in a particular area when zoomed in. The application is accessible online at \texttt{http://shiny.parisgeo.cnrs.fr/transportation}. A screenshot of the interface is presented in Figure~\ref{fig:fig-1}. Main conclusion from interactive data exploration is that strong spatial and temporal heterogeneity is the rule. The temporal pattern recurring most often, peak and off-peak hours is on a non-negligible proportion of days perturbed. In a first approximation, non-peak hours may be approximated by a local stationary distribution of flows, whereas peaks are too narrow to allow the validation of the equilibrium assumption. Spatially we can observe that no spatial pattern is clearly emerging. It means that in case of a validity of static user equilibrium, meta-parameters ruling its establishment must vary at time scales smaller than one day. We argue that traffic system must in contrary be far-from-equilibrium, especially during peak hours when critical phase transitions occur at the origin of traffic jams. 



%%%%%%%%%%%%%%%%%%
\begin{figure}
\vspace{1cm}
\centering
\includegraphics[width=\textwidth]{gr1}
\caption{Capture of the web-application to explore spatio-temporal traffic data for Parisian region. It is possible to select date and time (precision of 15min on one month, reduced from initial dataset for performance purposes). A plot summarizes congestion patterns on the current day.}
\label{fig:fig-1}
\end{figure}
%%%%%%%%%%%%%%%%%%



%%%%%%%%%%%%%%%%%%%%%%%%
\subsection{Spatio-temporal Variability of Travel Path}


Following interactive exploration of data, we propose to quantify the spatial variability of congestion patterns to validate or invalidate the intuition that if equilibrium does exist in time, it is strongly dependent on space and localized. The variability in time and space of travel-time shortest paths is a first way to investigate flow stationarities from a game-theoretic point of view. Indeed, the static User Equilibrium is the stationary distribution of flows under which no user can improve its travel time by changing its route. A strong spatial variability of shortest paths at short time scales is thus evidence of non-stationarity, since a similar user will take a few time after a totally different route and not contribute to the same flow as a previous user. Such a variability is indeed observed on a non-negligible number of paths on each day of the dataset. We show in Figure~\ref{fig:fig-2} an example of extreme spatial variation of shortest path for a particular Origin-Destination pair.


The systematic exploration of travel time variability across the whole dataset, and associated travel distance, confirms, as described in Figure 3, that travel time absolute variability has often high values of its maximum across OD pairs, up to 25 minutes with a temporal local mean around 10min. Corresponding spatial variability produces detours up to 35km.


%%%%%%%%%%%%%%%%%%%
\begin{figure}
\centering
\vspace{1.5cm}
\includegraphics[width=0.47\textwidth]{gr21}\hfill
\includegraphics[width=0.47\textwidth]{gr22}
\caption{Spatial variability of travel-time shortest path (shortest path trajectory in dotted blue). In an interval of only 10 minutes, between 11/02/2016 00:06 (left) and 11/02/2016 00:16 (right), the shortest path between \emph{Porte d'Auteuil} (West) and \emph{Porte de Bagnolet} (East), increases in effective distance of $\simeq 37$km (with an increase in travel time of only 6min), due to a strong disruption on the ring of Paris.} 
\label{fig:fig-2}
\end{figure}
%%%%%%%%%%%%%%%%%%%



%%%%%%%%%%%%%%%%%%%
\begin{figure}[t]\vspace*{4pt}
\centering
\centerline{\includegraphics[width=0.8\textwidth]{gr31}}
\centerline{\includegraphics[width=0.8\textwidth]{gr32}}
\caption{Travel time (top) in min and corresponding travel distance (bottom) maximal variability on a two weeks sample. We plot the maximal on all OD pairs of the absolute variability between two consecutive time steps. Peak hours imply a high time travel variability up to 25 minutes and a path length variability up to 35km.}
\label{fig:fig-3}
\end{figure}
%%%%%%%%%%%%%%%%%%%




%%%%%%%%%%%%%%%%%%%
\subsection{Stability of Network measures}

The variability of potential trajectories observed in the previous section can be confirmed by studying the variability of network properties. In particular, network topological measures capture global patterns of a transportation network. Centrality and node connectivity measures are classical indicators in transportation network description as recalled in~\cite{bavoux2005geographie}. The transportation literature has developed elaborated and operational network measures, such as network robustness measures to identify critical links and measure overall network resilience to disruptions (an example among many is the Network Trip Robustness index introduced in~\cite{sullivan2010identifying}).


More precisely, we study the betweenness centrality of the transportation network, defined for a node as the number of shortest paths going through the node, i.e. by the equation

%%%%%%%%%%%%%%%
% equation betweeness
\begin{equation}
b_i = \frac{1}{N(N-1)}\cdot \sum_{o\neq d \in V}\mathbbm{1}_{i\in p(o\rightarrow d)}
\end{equation}
%%%%%%%%%%%%%%%

where $V$ is the set of network vertices of size $N$, and $p(o\rightarrow d)$ is the set of nodes on the shortest path between vertices o and d (the shortest path being computed with effective travel times). This index is more relevant to our purpose than other measures of centrality such as closeness centrality that does not include potential congestion as betweenness centrality does.




We show in Figure 4 the relative absolute variation of maximal betweenness centrality for the same time window than previous empirical indicators. More precisely we plot the value of

%%%%%%%%%%%%%%%
% eq relative variability
\begin{equation}
\Delta b(t) = \frac{\left|\max_i (b_i(t + \Delta t)) - \max_i (b_i(t))\right|}{\max_i (b_i(t))}
\end{equation}
%%%%%%%%%%%%%%%



where $\Delta t$ is the time step of the dataset (the smallest time window on which we can capture variability). This absolute relative variation has a direct meaning : a variation of 20\% (which is attained a significant number of times as shown in Fig.~\ref{fig:fig-4}) means that in case of a negative variation, at least this proportion of potential travels have changed route and the local potential congestion has decrease of the same proportion. In the case of a positive variation, a single node has captured at least 20\% of travels. Under the assumption (that we do not try to verify in this work and assume to be also not verified as shown by~\cite{zhu2010people}, but that we use as a tool to give an idea of the concrete meaning of betweenness variability) that users rationally take the shortest path and assuming that a majority of travels are realized such a variation in centrality imply a similar variation in effective flows, leading to the conclusion that they can not be stationary in time (at least at a scale larger than $\Delta t$) nor in space.


%%%%%%%%%%%%%%%%%%%
\begin{figure}
\includegraphics[width=\textwidth]{gr4}
\caption{Temporal stability of maximal betweenness centrality. We plot in time the normalized derivative of maximal betweenness centrality, that expresses its relative variations at each time step. The maximal value up to 25\% correspond to very strong network disruption on the concerned link, as it means that at least this proportion of travelers assumed to take this link in previous conditions should take a totally different path.}
\label{fig:fig-4}
\end{figure}
%%%%%%%%%%%%%%%%%%%





%%%%%%%%%%%%%%%%%%%
\subsection{Spatial heterogeneity of equilibrium}


To obtain a different insight into spatial variability of congestion patterns, we propose to use an index of spatial autocorrelation, the Moran index (defined e.g. in~\cite{tsai2005quantifying}). More generally used in spatial analysis with diverse applications from the study of urban form to the quantification of segregation, it can be applied to any spatial variable. It allows to establish neighborhood relations and unveils spatial local consistence of an equilibrium if applied on localized traffic variable. At a given point in space, local autocorrelation for variable c is computed by

%%%%%%%%%%%%
% Moran index def
\begin{equation}
\rho_i = \frac{1}{K}\cdot \sum_{i\neq j}{w_{ij}\cdot (c_i - \bar{c})(c_j - \bar{c})}
\end{equation}
%%%%%%%%%%%%

where $K$ is a normalization constant equal to the sum of spatial weights times variable variance and $\bar{c}$ is variable mean. In our case, we take spatial weights of the form $w_{ij} = \exp{\left(\frac{-d_{ij}}{d_0}\right)}$ with $d_0$ typical decay distance and compute the autocorrelation of link congestion localized at link center. We capture therefore spatial correlations within a radius of same order than decay distance around the point $i$. The mean on all points yields spatial autocorrelation index $I$. A stationarity in flows should yield some temporal stability of the index.


Figure~\ref{fig:fig-5} presents temporal evolution of spatial autocorrelation for congestion. As expected, we have a strong decrease of autocorrelation with distance decay parameter, for both amplitude and temporal average. The high temporal variability implies short time scales for potential stationarity windows. When comparing with congestion (fitted to plot scale for readability) for 1km decay, we observe that high correlations coincide with off-peak hours, whereas peaks involve vanishing correlations. Our interpretation, combined with the observed variability of spatial patterns, is that peak hours correspond to chaotic behaviour of the system, as jams can emerge in any link: correlation thus vanishes as feasible phase space for a chaotic dynamical system is filled by trajectories in an uniform way what is equivalent to apparently independent random relative speeds.


%%%%%%%%%%%%%%%%
\begin{figure}
% Spatial 
\includegraphics[width=\textwidth,height=0.6\textheight]{gr5}
\caption{Spatial auto-correlations for relative travel speed on two weeks. We plot for varying value of decay parameter (1,10km) values of auto-correlation index in time. Intermediate values of decay parameter yield a rather continuous deformation between the two curves. Points are smoothed with a 2h span to ease reading. Vertical dotted lines correspond to midnight each day. Purple curve is relative speed fitted at scale to have a correspondence between auto-correlation variations and peak hours.}
\label{fig:fig-5}
\end{figure}
%%%%%%%%%%%%%%%%




%%%%%%%%%%%%%%%%%%%%
\section{Discussion}

\subsection{Theoretical and practical implications of empirical conclusions}

We argue that the theoretical implications of our empirical findings do not imply in a total discarding of the Static User Equilibrium framework, but unveil more a need of stronger connections between theoretical literature and empirical studies. If each newly introduced theoretical framework is generally tested on one on more case study, there are no systematic comparisons of each on large and different datasets and on various objectives (prediction of traffic, reproduction of stylized facts, etc.) as systematic reviews are the rule in therapeutic evaluation for example. This imply however broader data and model sharing practices than the current ones. The precise knowledge of application potentialities for a given framework may induce unexpected developments such as its integration into larger models. The example of Land-use and Transportation Interaction studies (LUTI models) is a good illustration of how the SUE can still be used for larger purpose than transportation modeling. \cite{kryvobokov2013comparison} describe two LUTI models, one of which includes two equilibria for four-step transportation model and for land-use evolution (households and firms relocation), the other being more dynamical. The conclusion is that each model has its own advantages regarding the pursued objective, and that the static model can be used for long time policy purposes, whereas the dynamic model provide more precise information at smaller time scale. In the first case, a more complicated transportation module would have been complicated to include, what is an advantage of the static user equilibrium.


Concerning practical applications, it seems natural that static models should not be used for traffic forecast and management at small time scales (week or day) and efforts should be made to implement more realistic models. However the use of models by the planning and engineering community is not necessarily directly related to academic concerns and state-of-the-art. For the particular case of France and mobility models, \cite{commenges2013invention} showed that engineers had gone to the point of constructing inexistent problems and implementing corresponding models that they had imported from a totally different geographical context (planning in the United States). The use of one framework or type of model has historical reasons that may be difficult to overcome.



\subsection{Towards explanative interpretations of non-stationarity}


An assumption we formulate regarding the origin of non-stationarity of network flows, in view of data exploration and quantitative analysis of the database, is that the network is at least half of the time highly congested and in a critical state. The off-peak hours are the larger potential time windows of spatial and temporal stationarity, but consist in less than half of the time. As already interpreted through the behavior of autocorrelation indicator, a chaotic behavior may be at the origin of such variability in the congested hours. The same way a supercritical fluid may condense under the smallest external perturbation, the state of the link may qualitatively change with a small incident, producing a network disruption that may propagate and even amplify. The direct effect of traffic events (notified incidents or accidents) can not be studied without external data, and it could be interesting to enrich the database in that direction. It would allow establishing the proportion of disruptions that do appear to have a direct effect and quantify a level of criticality of network congestion in time, or to investigate more precise effects such as the consequences of an incident on traffic of the opposite lane.




\subsection{Possible developments}

Further work may be planned towards a more refined assessment of temporal stability on a region of the network, i.e. the quantitative investigation of consideration of peak stationarity given above. To do so we propose to compute numerically Liapounov stability of the dynamical system ruling traffic flows using numerical algorithms such as described by~\cite{goldhirsch1987stability}. The value of Liapounov exponents provides the time scale by which the unstable system runs out of equilibrium. Its comparison with peak duration and average travel time, across different spatial regions and scales should provide more information on the possible validity of the local stationarity assumption. This technique has already been introduced at an other scale in transportation studies, as e.g.~\cite{tordeux2016jam} that study the stability of speed regulation models at the microscopic scale to avoid traffic jams.



Other research directions may consist in the test of other assumptions of static user equilibrium (as the rational shortest path choice, which would be however difficult to test on such an aggregated dataset, implying the use of simulation models calibrated and cross-validated on the dataset to compare assumptions, without necessarily a direct clear validation or invalidation of the assumption), or the empirical computation of parameters in stochastic or dynamical user equilibrium frameworks. 



\section{Conclusion}


We have described an empirical study aimed at a simple but from our point of view necessary investigation of the existence of the static user equilibrium, more precisely of its stationarity in space and time on a metropolitan highway network. We constructed by data collection a traffic congestion dataset for the highway network of Greater Paris on 3 months with two minutes temporal granularity. The interactive exploration of the dataset with a web application allowing spatio-temporal data visualization helped to guide quantitative studies. Spatio-temporal variability of shortest paths and of network topology, in particular betweenness centrality, revealed that stationarity assumptions do not hold in general, what was confirmed by the study of spatial autocorrelation of network congestion. We suggest that our findings highlight a general need of higher connections between theoretical and empirical studies, as our work can discard misunderstandings on the theoretical static user equilibrium framework and guide the choice of potential applications.







%% References
%%
%% Following citation commands can be used in the body text:
%% Usage of \cite is as follows:
%%   \cite{key}         ==>>  [#]
%%   \cite[chap. 2]{key} ==>> [#, chap. 2]
%%

%The citation must be used in following style: \cite{article-minimal} \cite{article-full} \cite{article-crossref} \cite{whole-journal}.
%% References with BibTeX database:

\bibliography{biblio}
\bibliographystyle{elsarticle-harv}





%%%%%%%%%%%%%%%%%%%%
% Original templates







%Here introduce the paper, and put a nome¬nclature if necessary, in a box with the same font size as the rest of the paper. The paragraphs continue from here and are only separated by headings, subheadings, images and formulae. The section headings are arranged by numbers, bold and 10 pt. Here follows further instructions for authors.

%\begin{nomenclature}
%\begin{deflist}[A]
%\defitem{A}\defterm{radius of}
%\defitem{B}\defterm{position of}
%\defitem{C}\defterm{further nomenclature continues down the page inside the text box\vspace*{-8pt}}
%\end{deflist}
%\end{nomenclature}
%\vspace*{0pt}

%\subsection{ Structure}
%Files must be in LaTeX format only and should be formatted for direct printing, using the CRC LaTeX template provided. Figures and tables should be embedded and not supplied separately. 

%Please make sure that you use as much as possible normal fonts in your documents. Special fonts, such as fonts used in the Far East (Japanese, Chinese, Korean, etc.) may cause problems during processing. To avoid unnecessary errors you are strongly advised to use the `spellchecker' function of TeX Editor. Follow this order when typing manuscripts: Title, Authors, Affiliations, Abstract, Keywords, Main text (including figures and tables), Acknowledgements, References, Appendix. Collate acknowledgements in a separate section at the end of the article and do not include them on the title page, as a footnote to the title or otherwise.

%Bulleted lists may be included and should look like this:
%\begin{itemize}[]
%\item First point
%\item Second point
%\item And so on
%\end{itemize}

%Ensure that you return to the `Els-body-text' style, the style that you will mainly be using for large blocks of text, when you have completed your bulleted list. 

%Please do not alter the formatting and style layouts which have been set up in this template document. As indicated in the template, papers should be prepared in single column format suitable for direct printing onto paper with trim size $192 \times 262$ mm. Do not number pages on the front, as page numbers will be added separately for the preprints and the Proceedings. Leave a line clear between paragraphs. All the required style templates are provided in the file ``LaTeX Template'' with the appropriate name supplied, e.g. choose 1. Els1st-order-head for your first order heading text, els-abstract-text for the abstract text etc.

%\subsection{ Tables}

%All tables should be numbered with Arabic numerals. Every table should have a caption. Headings should be placed above tables, left justified. Only horizontal lines should be used within a table, to distinguish the column headings from the body of the table, and immediately above and below the table. Tables must be embedded into the text and not supplied separately. Below is an example which the authors may find useful.

%\begin{table}[h]
%\caption{An example of a table.}
%\begin{tabular*}{\hsize}{@{\extracolsep{\fill}}lll@{}}
%\toprule
%An example of a column heading & Column A ({\it{t}}) & Column B ({\it{t}})\\
%\colrule
%And an entry &   1 &  2\\
%And another entry  & 3 &  4\\
%And another entry &  5 &  6\\
%\botrule
%\end{tabular*}
%\end{table}










%%\enlargethispage{12pt}

%\subsection{ Construction of references}

%References must be listed at the end of the paper. Do not begin them on a new page unless this is absolutely necessary. Authors should ensure that every reference in the text appears in the list of references and vice versa. Indicate references by \cite{clark} or \cite{Deal} or \cite{Fachinger2006} in the text. 

%Some examples of how your references should be listed are given at the end of this template in the `References' section, which will allow you to assemble your reference list according to the correct format and font size.

%Reference generation by using bibliography style commands for LaTeX template only.

%The author may use ``elsarticle-harv.bst'' as per the style required in document. The sample bib file could be referred. 
%If the author may using bibstyle for providing references author must comment the bibliography section in TeX file, Bibtex will generate the reference automatically.

%If the author may not able to view the references in output same could be done by copying the bibliography section from ``filename.bbl'' file and paste in TeX file.







%\subsection{Section headings}
%Section headings should be left justified, bold, with the first letter capitalized and numbered consecutively, starting with the Introduction. Sub-section headings should be in capital and lower-case italic letters, numbered 1.1, 1.2, etc,~and left justified, with second~and subsequent lines indented. All headings should have a minimum of two text lines after them before a page or column break.
%Ensure the text area is not blank except for the last page.

%\subsection{General guidelines for the preparation of your text}
%Avoid hyphenation at the end of a line. Symbols denoting vectors and matrices should be indicated in bold type. Scalar variable names should normally be expressed using italics. Weights and measures should be expressed in SI units. All non-standard abbreviations or symbols must be defined when first mentioned, or a glossary provided.

%\subsection{File naming and delivery}
%Please title your files in this order `procedia acronym\_conference acronym\_authorslastname'.  Submit both the source file and the PDF to the Guest Editor.

%Artwork filenames should comply with the syntax ``aabbbbbb.ccc'', where:\vspace*{-12pt}
%\begin{itemize}
%\item a = artwork component type
%\item b = manuscript reference code
%\item c = standard file extension

%Component types:
%\item gr = figure
%\item pl = plate
%\item sc = scheme
%\item fx = fixed graphic
%\end{itemize}


%\subsection{Footnotes}
%Footnotes should be avoided if possible. Necessary footnotes should be denoted in the text by consecutive superscript letters\footnote{Footnote text.}. The footnotes should be typed single spaced, and in smaller type size (8 pt), at the foot of the page in which they are mentioned, and separated from the main text by a one line space extending at the foot of the column. The `Els-footnote' style is available in the ``TeX Template'' for the text of the footnote.

%Please do not change the margins of the template as this can result in the footnote falling outside printing range.


%\section{Illustrations}
%All figures should be numbered with Arabic numerals (1,2,3,\,$\ldots.$). Every figure should have a caption. All\break photographs, schemas, graphs and diagrams are to be referred to as figures. Line drawings should be good quality\break scans or true electronic output. Low-quality scans are not acceptable. Figures must be embedded into the text and not supplied separately. In MS word input the figures must be properly coded. Preferred format of figures are PNG, JPEG, GIF etc. Lettering and symbols should be clearly defined either in the caption or in a legend provided as part of the figure. Figures should be placed at the top or bottom of a page wherever possible, as close as possible to the first reference to them in the paper. Please ensure that all the figures are of 300 DPI resolutions as this will facilitate good output.
%\begin{figure}[t]\vspace*{4pt}
%\centerline{\includegraphics{fx1}\hspace*{5mm}\includegraphics{fx1}}
%\centerline{\includegraphics{gr1}}
%\caption{(a) first picture; (b) second picture.}
%\end{figure}

%The figure number and caption should be typed below the illustration in 8 pt and left justified [{{\bfseries\itshape Note:}} one-line captions of length less than column width (or full typesetting width or oblong) centered]. For more guidelines and information to help you submit high quality artwork please visit: http://www.elsevier.com/artworkinstructions\break Artwork has no text along the side of it in the main body of the text. However, if two images fit next to each other, these may be placed next to each other to save space. For example, see Fig.~1. 


%\section{Equations}
%Equations and formulae should be typed in Mathtype, and numbered consecutively with Arabic numerals in parentheses on the right hand side of the page (if referred to explicitly in the text). They should also be separated from the surrounding text by one space
%\begin{equation}
%\begin{array}{lcl}
%\displaystyle X_r &=& \displaystyle\dot{Q}^{''}_{rad}\left/\left(\dot{Q}^{''}_{rad} + \dot{Q}^{''}_{conv}\right)\right.\\[6pt]
%\displaystyle \rho &=& \displaystyle\frac{\vec{E}}{J_c(T={\rm const.})\cdot\left(P\cdot\left(\displaystyle\frac{\vec{E}}{E_c}\right)^m+(1-P)\right)}
%\end{array}
%\end{equation}


%\section{Online license transfer}
%All authors are required to complete the Procedia exclusive license transfer agreement before the article can be published, which they can do online. This transfer agreement enables Elsevier to protect the copyrighted material for the authors, but does not relinquish the authors' proprietary rights. The copyright transfer covers the exclusive rights to reproduce and distribute the article, including reprints, photographic reproductions, microfilm or any other reproductions of similar nature and translations. Authors are responsible for obtaining from the copyright holder, the permission to reproduce any figures for which copyright exists.








%\section*{Acknowledgements}

%Acknowledgements and Reference heading should be left justified, bold, with the first letter capitalized but have no numbers. Text below continues as normal.

%% The Appendices part is started with the command \appendix;
%% appendix sections are then done as normal sections
%% \appendix

%% \section{}
%% \label{}

%\appendix
%\section{An example appendix}
%Authors including an appendix section should do so before References section. Multiple appendices should all have headings in the style used above. They will automatically be ordered A, B, C etc.

%\subsection{Example of a sub-heading within an appendix}
%There is also the option to include a subheading within the Appendix if you wish.




%% Authors are advised to use a BibTeX database file for their reference list.
%% The provided style file elsarticle-num.bst formats references in the required Procedia style

%% For references without a BibTeX database:

% \begin{thebibliography}{}

%% \bibitem must have the following form:
%%   \bibitem{key}...
%%

%\bibitem[Clark et al.(1962)]{clark}Clark, T., Woodley, R., De Halas, D., 1962. Gas-Graphite Systems, in ``{\it Nuclear Graphite}''. 
%In: Nightingale, R. (Ed.). Academic Press, New York, pp. 387.

%\bibitem[Deal and Grove(2009) ]{Deal}Deal, B., Grove, A., 1965. General Relationship for the Thermal Oxidation of Silicon. Journal of Applied Physics 36, 37--70.

%\bibitem[Deep(2009)]{Deep}Deep-Burn Project: Annual Report for 2009, Idaho National Laboratory, Sept. 2009.

%\bibitem[Fachinger(2004)]{Fachinger2004}Fachinger, J., den Exter, M., Grambow, B., Holgerson, S., Landesmann, C., Titov, M., Podruhzina, T., 2004. ``Behavior of spent HTR fuel elements in aquatic phases of repository host rock formations,'' 2nd International Topical Meeting on High Temperature Reactor Technology. Beijing, China, paper \#B08. 

%\bibitem[Fachinger(2006)]{Fachinger2006}Fachinger, J., 2006. Behavior of HTR Fuel Elements in Aquatic Phases of Repository Host Rock Formations. Nuclear Engineering \& Design 236,     54.


 %\end{thebibliography}















%\clearpage

%%%% This page is for instructions only, once the article is finalize please omit the below text before creating the final PDF
%\normalMode

%\section*{Instructions to Authors for LaTeX template:}

%\section{ZIP mode for LaTeX template:}

%The zip package is created as per the guide lines present on the URL http://www.elsevier.com/author-schemas/ preparing-crc-journal-articles-with-latex for creating the LaTeX zip file of Procedia LaTeX template.  The zip generally contains the following files:
%\begin{Itemize}[]\leftskip-17.7pt\labelsep3.3pt
%\item ecrc.sty
%\item  elsarticle.cls
%\item elsdoc.pdf
%\item .bst file
%\item Manuscript templates for use with these bibliographic styles
%\item  Generic and journal specific logos, etc.
%\end{Itemize}

%The LaTeX package is the main LaTeX template. All LaTeX support files are required for LaTeX pdf generation from the LaTeX template package. 

%{\bf Reference style .bst file used for collaboration support:} In the LaTeX template packages of all Procedia titles a new ``.bst'' file is used which supports collaborations downloaded from the path http://www.elsevier.com/author-schemas/the-elsarticle-latex-document-class

%\section{Reference styles used in  Procedia master templates:}
%\let\footnotesize\normalsize
%\hspace*{-10pt}\begin{tabular*}{\hsize}{@{}ll@{}}
%{\bf Title}&{\bf Reference style} \\[6pt]
%AASPRO  & 2 Harvard\\
%AASRI Procedia  & 3 Vancouver Numbered\\
%APCBEE Procedia  & 3 Vancouver Numbered\\
%EGYPRO  & 3 Vancouver Numbered\\
%FINE    & 2 Harvard\\
%IERI Procedia  & 3 Vancouver Numbered\\
%MATPR  & 1a Numbered without article titles\\
%MSPRO  & 2 Harvard\\
%PHPRO  & 2 Harvard\\
%PIUTAM  & 3a Embellished Vancouver \\
%Procedia CIRP  & 3 Vancouver Numbered\\
%PROCHE  & 3a Embellished Vancouver \\
%PROCS  & 3a Embellished Vancouver \\
%PROENG  & 1 Numbered\\
%PROENV  & 3a Embellished Vancouver \\
%PROEPS  & 3a Embellished Vancouver \\
%PROFOO    & 3a Embellished Vancouver \\
%PROMFG  & 1a Numbered without article titles\\
%PROTCY  & 3 Vancouver Numbered\\
%PROVAC  & 3a Embellished Vancouver \\
%SBSPRO  & 5 APA\\
%SEPRO  & 3a Embellished Vancouver \\
%AQPRO & 2 Harvard\\
%UMKPRO & 5 APA\\
%TRPRO  & 2 Harvard\\
%\end{tabular*}









\end{document}

%%
%% End of file `procs-template.tex'.
